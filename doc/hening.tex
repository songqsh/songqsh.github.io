\documentclass[11pt]{amsart}
\usepackage{geometry}                % See geometry.pdf to learn the layout options. There are lots.
\geometry{letterpaper}                   % ... or a4paper or a5paper or ... 
%\geometry{landscape}                % Activate for for rotated page geometry
%\usepackage[parfill]{parskip}    % Activate to begin paragraphs with an empty line rather than an indent
\usepackage{graphicx}
\usepackage{amssymb}
\usepackage{epstopdf}
\DeclareGraphicsRule{.tif}{png}{.png}{`convert #1 `dirname #1`/`basename #1 .tif`.png}

\title{Brief Article}
\author{The Author}
%\date{}                                           % Activate to display a given date or no date

\begin{document}
%\maketitle
%\section{}
%\subsection{}

\noindent {\bf Title:}
The Inverse First Passage Time Problem for killed Brownian motion
\vspace{.2in}

\noindent
{\bf Speaker:} Alex Hening from Tufts
\vspace{.2in}



\noindent
{\bf Abstract:}
The inverse first passage time problem asks whether, for a Brownian motion $B$ and a nonnegative random variable $\zeta$, there exists a time-varying barrier $b$ such that $\mathbb{P}\{B_s > b(s), \, 0 \le s \le t\} = \mathbb{P}\{\zeta > t\}$. We study a "smoothed" version of this problem and ask whether there is a "barrier" $b$ such that $\mathbb{E}[\exp(-\lambda \int_0^t \psi(B_s - b(s)) \, ds)] = \mathbb{P}\{\zeta > t\}$, where $\lambda$ is a killing rate parameter and $\psi: \mathbb{R} \to [0,1]$ is a non-increasing function. We prove that if $\psi$ is suitably smooth, the function $t \mapsto \mathbb{P}\{\zeta > t\}$ is twice continuously differentiable, and the condition $0 < -\frac{d \log \mathbb{P}\{\zeta > t\}}{dt} < \lambda$ holds for the hazard rate of $\zeta$, then there exists a unique continuously differentiable function $b$ solving the smoothed problem. We show how this result leads to flexible models of default for which it is possible to compute expected values of contingent claims.


\end{document}  